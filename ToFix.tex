\documentclass{article}
\usepackage[utf8]{inputenc}
\usepackage[russian]{babel}
\usepackage{amsthm}
\usepackage{amsfonts}
\usepackage{mdwlist}
\usepackage{ulem}
\usepackage[a4paper, hmargin={2cm, 1.5cm}, vmargin={1.5cm, 2.0cm}]{geometry}

\newtheorem*{fix}{Fix}


\begin{document}
  \begin{fix}[01-02]
    Там, конечно, противоположное событие, а не занятие.
  \end{fix}

  \begin{fix}[01-02]
    В определении сигма-алгебры, в 3-ем свойстве, конечно, следствие, а не стремление.
  \end{fix}

  \begin{fix}[04-09]
    Вверху, разумеется, веростность суммы \(A_i\), а не \(P_i\).
  \end{fix}

  \begin{fix}[11-20]
    Должно быть \(\sqrt(2\pi)\), а не \(\sqrt(2)\pi\).
  \end{fix}

  \begin{fix}[10-19]
    После второго равенства, разумеется, в числителе не \(n\), а \(n!\).
  \end{fix}

  \begin{fix}[14-24]
    В первом определении путаница с обозначениями, правильно так: \(F_X(t) = \ldots\)
  \end{fix}

  \begin{fix}[27-45]
    В утверждении: не нужно тупить. Читать нужно не «тогда существует матожидание, равное…», а «тогда если существует матожидание, то оно равно…».
  \end{fix}

  \begin{fix}[30-49]
    В следствии: после второго неравенства, разумеется, написано \(\sum\limits_{n=0}^{\infty}(X \ge n)\).
  \end{fix}
\end{document}